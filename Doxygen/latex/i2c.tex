\hypertarget{i2c_i2c-hardware}{}\section{hardware}\label{i2c_i2c-hardware}
In its simplest form, an i2c bus has one master and a number of slaves that are connected by two wires\+: S\+CL (clock) and S\+DA (data). Both lines are pulled up by pull-\/up resistor, and can (only) be pulled down by a connected chip (open-\/collector, hwlib pin type \hyperlink{classhwlib_1_1pin__oc}{pin\+\_\+oc}).



~\newline
 

\hypertarget{i2c_i2c-transaction-format}{}\section{transaction formats}\label{i2c_i2c-transaction-format}
An i2c transaction is either a read transaction or a write transaction. In both cases, the transaction starts with the master transmitting a control byte, which contains the address of the slave chip, and one bit that indicates whether it is a read or a write transaction. The bits of a byte are transferred M\+SB (most significant bit) first.





Next the slave chip receives (write transaction) or transmits (read transaction) as many bytes as the master asks for.



~\newline
 

\hypertarget{i2c_i2c-bits}{}\section{bit-\/level signaling}\label{i2c_i2c-bits}
At the bit level, master generates clock pulses by pulling the S\+CL line low. While the S\+CL is low, the master or slave can put a bit on the S\+DA line by pulling it down (for a 0) or letting it float (for a 1).

The S\+CL line is always driven by the master (unless the slave uses clock-\/stretching), the S\+DA line is driven by the device on the bus that sends the bit.



Two special conditions are used. To signal the start (S) of a transaction, the sda is pulled low while the clk is high. The reverse is used to indicate a stop (P, the end of a transaction)\+: the dta is released (goes high) while the clock is high.



~\newline
 

\hypertarget{i2c_i2c-addressing}{}\section{addressing within a slave}\label{i2c_i2c-addressing}
Simple slave chips that have only one data byte that can be read or written using a single-\/byte read or write transaction to read or write that data byte.

Slave chips that have more than one address that can be written often use a write transaction where the first byte(s) written determine the address (within the slave chip), and the subsequent byte(s) are written to that address (and to the next addresses).

An I2C read transaction addresses the slave chip, but has no provision to specify an address within the slave chip. The common trick is that a read addresses the last address specified by a (previous) write transaction. Hence to read from address X first a write is done to address X, but the transaction stops after the X. Hence nothing is written, but this sets the address pointer inside the slave chip. Now a read transaction reads from this address.

As always, consult the datasheet of the chip for the details!

~\newline
 

\hypertarget{i2c_i2c-ownership}{}\section{I2\+C ownership}\label{i2c_i2c-ownership}
The I2C bus was invented by Philips, who had a patent on it. Hence other manufacturers that implemented the I2C bus on their chips had either to pay royalties to Philips, or tried to avoid this by implementing a protocol that was compatible with I2C, without mentioning I2C. The I2C patent has expired, but you can still find many chips that are described as \textquotesingle{}implementing a two-\/wire protocol\textquotesingle{} or something similar. In most cases this means that the chip implements I2C.

references\+:
\begin{DoxyItemize}
\item \href{http://www.nxp.com/documents/user_manual/UM10204.pdf}{\tt I2C bus specification and user manual (pdf)}
\item \href{http://i2c.info/i2c-bus-specification}{\tt I2C Bus Specification (info page)}
\item \href{https://en.wikipedia.org/wiki/I2C}{\tt I2C Bus (wikipedia)}
\end{DoxyItemize}

~\newline
 

\hypertarget{i2c_i2c-bus}{}\section{i2c bus}\label{i2c_i2c-bus}
An i2c bus object represents an i2c bus. An i2c bus can be used (in order of preference)
\begin{DoxyItemize}
\item by creating a channel to a slave and using the operations of that channel
\item by read or write transactions or a transaction object, which specify the slave
\item through its primitives
\end{DoxyItemize}

\tabulinesep=1mm
\begin{longtabu} spread 0pt [c]{*{2}{|X[-1]}|}
\hline
\rowcolor{\tableheadbgcolor}\textbf{ i2c bus attributes and functions }&\textbf{ use  }\\\cline{1-2}
\endfirsthead
\hline
\endfoot
\hline
\rowcolor{\tableheadbgcolor}\textbf{ i2c bus attributes and functions }&\textbf{ use  }\\\cline{1-2}
\endhead
channel( s ) &creates a channel to slave s \\\cline{1-2}
\hyperlink{classhwlib_1_1i2c__bus_ac4dd659c140ae5f8f7d6bbd48351d88f}{read( s, d )} &performs a single byte read transaction \\\cline{1-2}
\hyperlink{classhwlib_1_1i2c__bus_ac4dd659c140ae5f8f7d6bbd48351d88f}{read( s, d, n )} &performs an n byte read transaction \\\cline{1-2}
\hyperlink{classhwlib_1_1i2c__bus_ad0b808c4d9b1ed7b16ee13c735c25597}{write( s, d )} &performs a single byte write transaction \\\cline{1-2}
\hyperlink{classhwlib_1_1i2c__bus_ad0b808c4d9b1ed7b16ee13c735c25597}{write( s, d, n )} &performs an n byte write transaction \\\cline{1-2}
transactions.read( s ) &creates a read transaction \\\cline{1-2}
transactions.write( s ) &creates a write transaction \\\cline{1-2}
\hyperlink{classhwlib_1_1i2c__primitives}{primitives} &low-\/level operations \\\cline{1-2}
\end{longtabu}
The read and write operations are easier to use than a transaction, but require the data to be read or written is one contiguous block.

A \hyperlink{classhwlib_1_1i2c__bus__bit__banged__scl__sda}{bit-\/banged i2c bus implementation} is provided. An object of this class is constructed from the two pins (scl and scd), which must be \hyperlink{classhwlib_1_1pin__oc}{pin\+\_\+oc} pins.

~\newline
 

\hypertarget{i2c_i2c-channel}{}\section{i2c channel}\label{i2c_i2c-channel}
An i2c channel abstracts the communication over an i2c bus to a specific slave. An i2c channel is created from an i2c bus by providing the slave address.

An i2c channel can be used (in order of preference)
\begin{DoxyItemize}
\item by read or write transactions on a single byte or a block of bytes
\item by creating a read or write transaction and using the operations of that transaction
\item through its primitives (which are same as the i2c bus primitives)
\end{DoxyItemize}

The read and write operations are easier to use than a transaction, but require the data to be read or written is one contiguous block.

\tabulinesep=1mm
\begin{longtabu} spread 0pt [c]{*{2}{|X[-1]}|}
\hline
\rowcolor{\tableheadbgcolor}\textbf{ i2c channel attributes and functions }&\textbf{ use  }\\\cline{1-2}
\endfirsthead
\hline
\endfoot
\hline
\rowcolor{\tableheadbgcolor}\textbf{ i2c channel attributes and functions }&\textbf{ use  }\\\cline{1-2}
\endhead
read( d ) &performs a single byte read transaction \\\cline{1-2}
read( d, n ) &performs an n byte read transaction \\\cline{1-2}
write( d ) &performs a single byte write transaction \\\cline{1-2}
write( d, n ) &performs an n byte write transaction \\\cline{1-2}
transactions.read() &creates a read transaction \\\cline{1-2}
transactions.write() &creates a write transaction \\\cline{1-2}
\hyperlink{classhwlib_1_1i2c__primitives}{primitives} &low-\/level operations \\\cline{1-2}
\end{longtabu}


~\newline
 

\hypertarget{i2c_i2c-transactions}{}\section{i2c transactions}\label{i2c_i2c-transactions}
An i2c transaction encompasses a complete read or write transaction. A transaction is created from an i2c\+\_\+bus object and the slave address. The transaction can be used to read (for a read transaction) or write (for a write transaction) data. A transaction allows for multiple read or write operations within a single transaction. A transaction is started (control byte send) by the transaction constructor. A transaction is closed (final N\+A\+CK send) by the transaction destructor.

\tabulinesep=1mm
\begin{longtabu} spread 0pt [c]{*{2}{|X[-1]}|}
\hline
\rowcolor{\tableheadbgcolor}\textbf{ i2c read transaction functions }&\textbf{ use  }\\\cline{1-2}
\endfirsthead
\hline
\endfoot
\hline
\rowcolor{\tableheadbgcolor}\textbf{ i2c read transaction functions }&\textbf{ use  }\\\cline{1-2}
\endhead
\hyperlink{classhwlib_1_1i2c__read__transaction_a404a7d9db0ddd18aaeac84b266e89f9e}{read\+\_\+byte()} &performs a single byte read \\\cline{1-2}
\hyperlink{classhwlib_1_1i2c__read__transaction_a4c886112d92dddb502cb18a18c3a385f}{read( d, n )} &performs an n byte read \\\cline{1-2}
\end{longtabu}
~\newline


\tabulinesep=1mm
\begin{longtabu} spread 0pt [c]{*{2}{|X[-1]}|}
\hline
\rowcolor{\tableheadbgcolor}\textbf{ i2c write transaction functions }&\textbf{ use  }\\\cline{1-2}
\endfirsthead
\hline
\endfoot
\hline
\rowcolor{\tableheadbgcolor}\textbf{ i2c write transaction functions }&\textbf{ use  }\\\cline{1-2}
\endhead
\hyperlink{classhwlib_1_1i2c__write__transaction_a15fdb954f92b784f2d723892b57f6728}{write( d )} &performs a single byte write \\\cline{1-2}
\hyperlink{classhwlib_1_1i2c__write__transaction_a15fdb954f92b784f2d723892b57f6728}{write(const uint8\+\_\+t data\mbox{[}\mbox{]},size\+\_\+t n)} &performs an n byte write \\\cline{1-2}
\end{longtabu}


~\newline
 

\hypertarget{i2c_i2c-primitives}{}\section{i2c primitives}\label{i2c_i2c-primitives}
The class i2c\+\_\+primitives abstracts the primitive i2c operations, from which the higher-\/level read and write transactions are implemented. These are the operations that must be implemented by an i2c implementation.

As an i2c user, you should probably never use these functions.

\tabulinesep=1mm
\begin{longtabu} spread 0pt [c]{*{2}{|X[-1]}|}
\hline
\rowcolor{\tableheadbgcolor}\textbf{ function }&\textbf{ use  }\\\cline{1-2}
\endfirsthead
\hline
\endfoot
\hline
\rowcolor{\tableheadbgcolor}\textbf{ function }&\textbf{ use  }\\\cline{1-2}
\endhead
\hyperlink{classhwlib_1_1i2c__primitives_adfa6c493163d397f58a3e2ba4617dbac}{hwlib\+::i2c\+\_\+primitives\+::read\+\_\+ack} read\+\_\+ack() &read and return an (expected) ack bit \\\cline{1-2}
\hyperlink{classhwlib_1_1i2c__primitives_a711df86e5129b7daeff7a622b7b734c6}{hwlib\+::i2c\+\_\+primitives\+::write\+\_\+ack} write\+\_\+ack() &write an ack bit \\\cline{1-2}
\hyperlink{classhwlib_1_1i2c__primitives_a5ca312553bc0817ffecf5f90caf96396}{hwlib\+::i2c\+\_\+primitives\+::write\+\_\+nack} write\+\_\+nack() &read and ack bit \\\cline{1-2}
\hyperlink{classhwlib_1_1i2c__primitives_aa5227ae39d6dd5957cf47b0b761b475b}{hwlib\+::i2c\+\_\+primitives\+::write(uint8\+\_\+t)} write( b ) &write one byte (8 bits) \\\cline{1-2}
\hyperlink{classhwlib_1_1i2c__primitives_a21c9843d3a7801781e576013a1e154f9}{hwlib\+::i2c\+\_\+primitives\+::read\+\_\+byte} read\+\_\+byte() &read one byte (8 bits) \\\cline{1-2}
\hyperlink{classhwlib_1_1i2c__primitives_aa5227ae39d6dd5957cf47b0b761b475b}{hwlib\+::i2c\+\_\+primitives\+::write} write(uint8\+\_\+t data\mbox{[}$\,$\mbox{]},size\+\_\+t n) &write n bytes (from d ) \\\cline{1-2}
\hyperlink{classhwlib_1_1i2c__primitives_a2a8cc988531ea774d39d726f213f8585}{hwlib\+::i2c\+\_\+primitives\+::read} read( d, n ) &read n bytes (into d) \\\cline{1-2}
\end{longtabu}


~\newline
 